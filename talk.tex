%% Style copied from https://github.com/omelkonian/presentations/tree/master/%5B2019.08.20%5D%20BitML%20(SRC%20Presentation%20%40%20ICFP))

\newif\iftalk
\newif\ifacm

\talktrue

%% \documentclass[aspectratio=43]{beamer}
\documentclass[aspectratio=169]{beamer}

\RequirePackage{ agda, tikz-cd, amssymb, catchfilebetweentags
               , newunicodechar , datetime2, geometry, relsize, xcolor
               , caption, subcaption, natbib }

\usetheme[
   block=fill,
   background=light,
   titleformat=smallcaps,
   progressbar=frametitle,
   numbering=none,
]{metropolis}%Warsaw,Madrid
\setbeamersize{text margin left=.5cm,text margin right=.5cm}

\usepackage[pdf]{graphviz}
\usepackage{tikz-cd}

%% \renewcommand\alert[1]{\textcolor{mLightBrown}{#1}}

\newcommand\nc\newcommand
\nc\rnc\renewcommand

%%%%%%%%%% AGDA ALIASES

\newcommand{\AP}{\AgdaPrimitive}
\newcommand{\APT}{\AgdaPrimitiveType}
\newcommand{\AK}{\AgdaKeyword}
\newcommand{\AM}{\AgdaModule}
\newcommand{\AS}{\AgdaSymbol}
\newcommand{\AStr}{\AgdaString}
\newcommand{\AN}{\AgdaNumber}
\newcommand{\AD}{\AgdaDatatype}
\newcommand{\AF}{\AgdaFunction}
\newcommand{\ARe}{\AgdaRecord}
\newcommand{\AFi}{\AgdaField}
\newcommand{\AB}{\AgdaBound}
\newcommand{\AIC}{\AgdaInductiveConstructor}
\newcommand{\AC}{\AgdaComment}
\newcommand{\AG}{\AgdaGeneralizable}
\newcommand{\APo}{\AgdaPostulate}
\newcommand{\AU}{\AgdaUnderscore}

\newcommand{\Set}{\mathbf{Set}}

\newcommand{\AR}{\AgdaRef}

%% Targets I've turned off
\newcommand\NoAgdaTarget[1]{}

\nc\uni[2]{\newunicodechar{#1}{\ensuremath{#2}}}

%% Remember to express horizontal movement in ems and vertical in exs and
%% nothing in pt, to adapt to font size changes.

\uni{⦃}{\{\hspace{-0.38em}\{}
\uni{⦄}{\}\hspace{-0.38em}\}}

\nc\sm[1]{\mathsf{#1}}
%% \uni{ℕ}{\sm{N}}
%% \uni{𝔹}{\sm{B}}
%% \uni{𝔽}{\sm{F}}

\usepackage{bm}
\uni{ℕ}{\mathbb{N}}
\uni{ℚ}{\mathbb{Q}}

\usepackage{mathrsfs}
\uni{𝒯}{\mathscr{T}}
%% \uni{𝒯}{\mathfrak{T}}
%% \uni{𝒯}{\mathcal{T}}

\uni{ℓ}{\ell}

\uni{⊗}{\otimes}
\uni{×}{\times}
\uni{∘}{\circ}
\uni{∙}{\bullet}
\uni{▵}{\vartriangle}
\uni{∎}{\blacksquare}

\usepackage{stmaryrd}

\uni{⇾}{\rightarrow}
\uni{⇨}{\rightarrowtriangle}
\uni{↝}{\leadsto}
\uni{↓}{\downarrow}
\uni{⇓}{\Downarrow}
\uni{↑}{\uparrow}

\uni{δ}{\delta}
\uni{λ}{\lambda}

\uni{≈}{\approx}
\uni{≡}{\equiv}
\uni{≗}{\circeq}
\uni{∃}{\exists}
\uni{∀}{\forall}
\uni{∈}{\in}
\uni{≤}{\le}
\uni{⊹}{\dotplus}
\uni{◇}{\diamond}
\uni{∅}{\varnothing}
\uni{∩}{\cap}
\uni{⊔}{\sqcup}
\uni{⊓}{\sqcap}
% \uni{⋯}{\cdots}
\uni{⋯}{\mathop{\cdot\cdot\cdot}}

\uni{∧}{\wedge}
\uni{∨}{\vee}
\uni{⊤}{\top}
\uni{⊥}{\bot}

%% Subscripts and superscripts
\nc\subs[1]{\ensuremath{_{\scriptscriptstyle #1}}}
\nc\sups[1]{\ensuremath{^{\scriptscriptstyle #1}}}

\uni{₀}{\subs 0}
\uni{₁}{\subs 1}
\uni{₂}{\subs 2}
\uni{₃}{\subs 3}
\uni{₄}{\subs 4}
\uni{₅}{\subs 5}
\uni{₆}{\subs 6}
\uni{₇}{\subs 7}
\uni{₈}{\subs 8}
\uni{₉}{\subs 9}

%% %%https://tex.stackexchange.com/questions/486120/using-the-unicode-prime-character-outside-math-mode-with-unicode-math-and-newuni
%% \AtBeginDocument{\newunicodechar{′}{\ensuremath{'}}}

\uni{′}{\ensuremath{'}}
\uni{″}{\ensuremath{'\hspace{-1pt}'}}

\usepackage{bbding}
\uni{✶}{\mbox{\tiny \SixStar}}

\uni{✢}{\raisebox{0.75pt}{\scalebox{0.65}{+}}}
\nc\carrot{\;\^{\;}}
\nc\inv{\ensuremath{^{-1}}}


%% To do: perhaps remove some substitutions below, and use the following.
\nc\SubStar[1]{\StrSubstitute{#1}{*}{\ast}}

\nc\ehat[1]{\ensuremath{\hat{#1}}}
\nc\edot[1]{\ensuremath{\dot{#1}}}
\nc\etilde[1]{\ensuremath{\tilde{#1}}}
%% \nc\etilde[1]{\~{#1}}

\usepackage{xstring}
\DeclareRobustCommand{\AgdaFormat}[2]{\IfStrEqCase{#1}{
  {\AU{}*\AU{}}{\_{\ast}\_}
  {0̇}{\edot{0}} {1̇}{\edot{1}}
  {+̇}{\edot{✢}}
  {*̇}{\edot{\ast}}
  {+}{✢}{\AU{}+̇\AU{}}{\_\edot{✢}\_}
  {*}{\ast}{\AU{}*̇\AU{}}{\_\edot{\ast}\_}
  {+̃}{\etilde{✢}}{\AU{}+̃\AU{}}{\_\etilde{✢}\_}
  {*̃}{\etilde{\ast}}{\AU{}*̃\AU{}}{\_\etilde{\ast}\_}
  {f̃}{\etilde{f}}{x̃}{\etilde{x}}
  {f̃⇃}{\etilde{\AFi{f}}}

{}{}}[#2]}


\nc\out[1]{}

%% \nc\noteOut[2]{\note{#1}\out{#2}}

%% To redefine for a non-draft
\nc\indraft[1]{#1}

%% I think \note gets defined by beamer.
\let\note\undefined

\definecolor{noteColor}{rgb}{0.5,0,0.3}

\nc\note[1]{\indraft{\textcolor{noteColor}{#1}}}

\nc\notefoot[1]{\note{\footnote{\note{#1}}}}

\nc\todo[1]{\note{To do: #1}}

\nc\eqnlabel[1]{\label{equation:#1}}
\nc\eqnref[1]{Equation~\ref{equation:#1}}
\nc\eqnreftwo[2]{Equations~\ref{equation:#1} and \ref{equation:#2}}

\nc\figlabel[1]{\label{fig:#1}}
\nc\figref[1]{Figure~\ref{fig:#1}}
\nc\figreftwo[2]{Figures~\ref{fig:#1} and \ref{fig:#2}}

\nc\seclabel[1]{\label{sec:#1}}
\nc\secref[1]{Section~\ref{sec:#1}}
\nc\secreftwo[2]{Sections~\ref{sec:#1} and~\ref{sec:#2}}
\nc\secrefs[2]{Sections \ref{sec:#1} through \ref{sec:#2}}

\nc\appref[1]{Appendix~\ref{sec:#1}}

%% The name \secdef is already taken
\nc\sectiondef[1]{\section{#1}\seclabel{#1}}
\nc\subsectiondef[1]{\subsection{#1}\seclabel{#1}}
\nc\subsubsectiondef[1]{\subsubsection{#1}\seclabel{#1}}

\nc\needcite{\note{[ref]}}

\nc\sectionl[1]{\section{#1}\seclabel{#1}}
\nc\subsectionl[1]{\subsection{#1}\seclabel{#1}}

\nc\workingHere{
\vspace{1ex}
\begin{center}
\setlength{\fboxsep}{3ex}
\setlength{\fboxrule}{4pt}
\huge\textcolor{red}{\framebox{Working here}}
\end{center}
\vspace{1ex}
}

%% For multiple footnotes at a point. Adapted to recognize \notefoot as well
%% as \footnote. See https://tex.stackexchange.com/a/71347,
\let\oldFootnote\footnote
\nc\nextToken\relax
\rnc\footnote[1]{%
    \oldFootnote{#1}\futurelet\nextToken\isFootnote}
\nc\footcomma[1]{\ifx#1\nextToken\textsuperscript{,}\fi}
\nc\isFootnote{%
    \footcomma\footnote
    \footcomma\notefoot
}

% Arguments: env, label, caption, body
\nc\figdefG[4]{\begin{#1}%[tbp]
\begin{center}
#4
\end{center}
\vspace{-1ex}
\caption{#3}
\figlabel{#2}
\end{#1}}

% Arguments: label, caption, body
\nc\figdef{\figdefG{figure}}
\nc\figdefwide{\figdefG{figure*}}

% Arguments: label, caption, body
\nc\figrefdef[3]{\figref{#1}\figdef{#1}{#2}{#3}}

\nc\picfile[1]{Figures/#1.pdf}

\nc\incpicW[2]{\includegraphics[width=#1\linewidth]{\picfile{#2}}}
\nc\incpic{\incpicW{0.7}}
%% \nc\incpic[1]{\includegraphics[width=0.7\linewidth]{\picfile{#1}}}

\nc\stdlibCitet[1]{\citet[\AM{#1}]{agda-stdlib}}
\nc\stdlibCitep[1]{\citep[\AM{#1}]{agda-stdlib}}
\nc\stdlibCite\stdlibCitep

%% https://tex.stackexchange.com/questions/94699/absolutely-definitely-preventing-page-break

\renewenvironment{samepage}
  {\par\nobreak\vfil\penalty0\vfilneg
   \vtop\bgroup}
  {\par\xdef\tpd{\the\prevdepth}\egroup
   \prevdepth=\tpd}

\nc\source{None}

\nc\twocolX[3]{\\[-1ex]
\begin{minipage}[c]{#1\linewidth}{#2}\end{minipage}
\begin{minipage}[c]{\linewidth}{#3}\end{minipage}\\[-1ex]}
\nc\twocol{\twocolX{0.45}}

\nc\threecolX[5]{\\[-1ex]
\begin{minipage}[c]{#1\linewidth}{#3}\end{minipage}
\begin{minipage}[c]{#2\linewidth}{#4}\end{minipage}
\begin{minipage}[c]{\linewidth}{#5}\end{minipage}\\[-1ex]}
\nc\threecol{\threecolX{0.3}{0.3}}

\nc\agda[1]{\ExecuteMetaData[\source.tex]{#1}}
\nc\agdaIn[2]{{\rnc\source{#1} \agda{#2}}}


\author{Atticus Kuhn}
\title{Denotationally Correct Computer Arithmetic}
\date{Early draft of \DTMnow{} GMT-8}



%% \usepackage{libertine}  %% [tt=false]

%% \setmathfont{XITSMath-Regular.otf}

%----------------------------------------------------------------------------

%% \titlegraphic{
%% \vspace*{7cm}
%% \includegraphics[keepaspectratio=true,height=1.4cm]{uu}
%% \hspace{1cm}
%% \includegraphics[keepaspectratio=true,height=1.4cm]{iohk}


\begin{document}

\begin{center}
\setbeamerfont{title}{size=\large}
\setbeamerfont{subtitle}{size=\small}
\maketitle
\setbeamerfont{title}{size=\Large}
\setbeamerfont{subtitle}{size=\large}
\end{center}
\section{Preliminaries}
\begin{frame}{About Denotational Design}

  \begin{defn}
    \textbf{Denotational Design} is a type of thinking that prioirtizes thinking about the meaning and creating precise and elegant specifications using tools from abstract algebra and category theory.
\end{defn}

\begin{notn}
  In denotational design, the function $\llbracket \cdot \rrbracket$ is used to take any object to its meaning.
\end{notn}
\end{frame}

\begin{frame}{Why Computer Arithmetic}
  Reasons why I chose computer arithmetic
  \begin{enumerate}
          \pause \item It is elementary; Most people have some exposure
          \pause \item It is a good way to show denotational design
    \end{enumerate}
    \pause The focus is not on any specific circuit component, but on specifications as to why it is \textbf{correct}
\end{frame}

\begin{frame}{Simplifications}
  For the purposes of this talk, I simplified from my paper:
  \begin{enumerate}
          \pause \item In my paper, I used the computer theorem prover language \textbf{Agda} to prove my propositions correct. You do not need to know programming for this talk.
          \pause \item In my paper, I talked about category theory, but for the sake of this talk, just imagine everything is occuring in the category of functions.
        \end{enumerate}
\end{frame}


\begin{frame}{Binary Basics}

We will represent binary numbers as lists of bits, where the least significant bit is on the left (little endian encoding).

As an additional preliminary, we expect the reader to be familiar with common bitwise operations, including $\cdot \oplus \cdot$, $\cdot \lor \cdot$, and $\cdot \land \cdot$ (see table~\ref{table:bitops}).
\begin{notn}
  We use $N$ to denote our number system, we use $\mathbb{B}$ to represent a bit,  and we use $\mathbb{B}^{n}$ to denote an $n$-bit representation.
\end{notn}
\begin{table}[h]
\centering
\begin{tabular}{||c c c||}
 \hline
 $\cdot \oplus \cdot$ & $\cdot \lor \cdot$ & $\cdot \land \cdot$  \\ [0.5ex]
 \hline\hline
 $0 \oplus 0 = 0$ & $0 \lor 0 = 0$ & $0 \land 0 = 0$ \\
 $0 \oplus 1 = 1$ & $0 \lor 1 = 1$ & $0 \land 1 = 0$ \\
 $1 \oplus 0 = 1$ & $1 \lor 0 = 1$ & $1 \land 0 = 0$ \\
 $1 \oplus 1 = 0$ & $1 \lor 0 = 1$ & $1 \land 1 = 1$ \\
 \hline
\end{tabular}
\caption{$\cdot \oplus \cdot$, $\cdot \lor \cdot$, and $\cdot \land \cdot$}
\label{table:bitops}
\end{table}



\end{frame}

\section{Addition}

\begin{frame}{Converting $\mathbb{B}^{n}$ to $N$ }
  Anything we do is only correct modulo our meaning function $\llbracket \cdot \rrbracket$.
  \begin{equation}
    \llbracket b_{0} b_{1} b_{2} \ldots b_{n-1} \rrbracket = \llbracket b_{0} \rrbracket + 2 \llbracket b_{1} \rrbracket + 4 \llbracket b_{2} \rrbracket + \cdots + 2^{n-1} \llbracket b_{n-1 } \rrbracket
\end{equation}
 \begin{figure}
   \digraph[width=\textwidth]{conversions}{
  node [shape=Mrecord]
  rankdir=LR
  input [label="{Input|{<in1>1|<in2>0|<in3>1}}"]
  add1 [label="{{<a11>1|<a12>4}|add|{<a1o>5}}"]
  dub1 [label="{{<dub11>2|<dub12>2}|add|{<dub1o>4}}"]
if1 [label="{{<if1i>1|1|0}|if|{<if1o>1}}"]
  output [label="{{<out>5}|output}"]
  if2 [label="{{<if2i>0|1|0}|if|{<if2o>0}}"]
  if3 [label="{{<if3i>1|1|0}|if|{<if3o>1}}"]
  add2 [label="{{<a21>0|<a22>2}|add|{<a2o>2}}"]
  add3 [label="{{<a31>1|<a32>0}|add|{<a3o>1}}"]
   dub2 [label="{{<dub21>1|<dub22>1}|add|{<dub2o>2}}"]
add1:a1o -> output:out [label="5"]
input:in1 -> if1:if1i  [label="1"]
if1:if1o -> add1:a11 [label="1"]
dub1:dub10 -> add1:a12 [label="3"]
input:in2 -> if2:if2i [label="0"]
input:in3 -> if3:if3i [label="1"]
if2:if2o -> add2:a21 [label="0"]
add2:a2o -> dub1:dub11 [label="2"]
add2:a2o -> dub1:dub12 [label="2"]
if3:if3o -> add3:a31 [label="1"]
add3:a3o -> dub2:dub22 [label="1"]
add3:a3o -> dub2:dub21 [label="1"]
dub2:dub2o -> add2:add22 [label="2"]
{rank=same; if1 if2 if3 }
   }
   \centering
   \caption{An Example showing $\llbracket 101 \rrbracket = 5$}
   \label{fig:conversionsbton}
\end{figure}

\end{frame}

\begin{frame}{Half-Adder Specification}

  A half adder is a function that adds two bits.
\begin{equation}
  \cdot +_{H} \cdot : \mathbb{B} \times \mathbb{B} \to \mathbb{B}^{2}
\end{equation}
  We need a correctness specification for a half-adder.
\pause
\begin{equation}\label{eqn:halfadder}
  \forall A,B \in \mathbb{B}^{1} \qquad \llbracket A +_{H} B \rrbracket = \llbracket A \rrbracket +_{N} \llbracket B \rrbracket
\end{equation}


\end{frame}

\begin{frame}{Half-Adder Example}
\begin{equation}\label{eqn:defhalfadd}
  \forall A,B\in \mathbb{B}^{1} \qquad A +_{H} B = (A \land B , A \oplus B)
\end{equation}
 \begin{figure}
   \digraph[width=\textwidth]{halfadder}{
  node [shape=Mrecord]
  rankdir=LR
  input [label="{Input|{<in1>1|<in2>1}}"]
  output [label="{{<out2>1|<out1>0}|output}"]
    xor [label="{{<xi1>1|<xi2>1} | xor | {<xo>0}}"]
    and [label="{{<ai1>1|<ai2>1} | and | {<ao>1}}"]
    input:in1 -> xor:xi1 [label="1"]
    input:in2 -> xor:xi2 [label="1"]
    input:in1 -> and:ai1 [label="1"]
    input:in2 -> and:ai2 [label="1"]
     xor:xo -> output:out1 [label="0"]
    and:ao -> output:out2 [label="1"]
   }
   \centering
   \caption{An Example Showing $1 +_{H} 1 = 10$}
\end{figure}



\end{frame}

\begin{frame}{Full-Adder Specification}
  A full-adder adds $3$ bits with possibly a carry.
  \begin{equation}
    +_{F}(\cdot , \cdot , \cdot) : \mathbb{B} \times \mathbb{B} \times \mathbb{B} \to \mathbb{B}^{2}
\end{equation}
\begin{equation}\label{eqn:fulladderspec}
  \forall A,B, C \in \mathbb{B}^{1} \qquad \llbracket +_{F}(A,B,C) \rrbracket  = \llbracket A \rrbracket  + \llbracket B \rrbracket + \llbracket C \rrbracket
\end{equation}

\end{frame}

\begin{frame}{Full-Adder Example}
\begin{equation}\label{eqn:fulladderdef}
  \forall A,B,C \in \mathbb{B}^{1} \qquad +_{F}(A,B,C) = (A \land B \lor (A \oplus B) \land C , A \oplus B \oplus C )
\end{equation}
 \begin{figure}
   \digraph[width=\textwidth]{fulladder}{
  node [shape=Mrecord]
  rankdir=LR
  input [label="{Input|{<in1>1|<in2>0|<in3>1}}"]
  output [label="{{<outc>1|<outs>0}|output}"]
    ad1 [label="{{<a11>1|<a12>0} | half-adder | {<a1c>0|<a1s>1}}"]
    ad2 [label="{{<a21>1|<a22>1} | half-adder | {<a2c>1|<a2s>0}}"]
    or [label="{{<oi1>0|<oi2>1} | or | {<oo>1}}"]
    input:in1 -> ad1:a11 [label="1"]
    input:in2 -> ad1:a12 [label="0"]
   input:in3 -> ad2:a22 [label="1"]
ad1:a1s -> ad2:a21 [label="1"]
ad2:a2s -> output:outs [label="0"]
ad2:a2c -> or:oi2 [label="1"]
ad1:a1c -> or:oi1 [label="0"]
or:oo -> output:outc [label="1"]
   }
   \centering
   \caption{An Example Showing $+_{F}(1,0,1) = 10$}
   \label{fig:fulladder}
\end{figure}

\end{frame}

\begin{frame}{Ripple Adder Specification}
  \begin{equation}
    \cdot +_{\mathbb{B}^{n}}^{\cdot} \cdot : \mathbb{B}^{1} \times \mathbb{B}^{n} \times \mathbb{B}^{n} \to \mathbb{B}^{n+1}
    \end{equation}

    \begin{equation}
      \forall A , B \in \mathbb{B}^{n} \quad \forall C \in \mathbb{B}^{1} \qquad \llbracket A +_{\mathbb{B}^{n}}^{C} B \rrbracket = \llbracket A \rrbracket +_{N} \llbracket B \rrbracket +_{N} \llbracket C \rrbracket
    \end{equation}
\end{frame}

\begin{frame}{Ripple Adder Specification}

\begin{table}
 \begin{tabular}{c|c|c|c}
     ${}^1$ & ${}^11$    &${}^1 0$    &1\\
     + & 1      & 1    & 1 \\ \hline
    1 & 1    & 0    &0\\ \hline
 \end{tabular}
 \centering
 \caption{Grade-School Addition}
 \label{tab:gradeschooladdition}
\end{table}
\begin{equation}\label{eqn:RCA}
  \begin{split}
  a_{n-1}\cdots a_{2}a_{1}a_{0} +_{\mathbb{B}^{n}}^{c_{0}} b_{n-1} \cdots b_{2}b_{1}b_{0} = (a_{n-1} \cdots a_{2}a_{1} +_{\mathbb{B}^{n-1}}^{c_{1}} b_{n-1} \cdots b_{2}b_{1})  r_{0} \\
  \text{where}\\
   c_{1}r_{0} = +_{F}(a_{0}, b_{0}, c_{0})
    \end{split}
  \end{equation}
\end{frame}
\begin{frame}{Ripple Adder Picture}
 \begin{figure}
   \digraph[width=\textwidth]{bitadder}{
node [shape=Mrecord]
  rankdir=LR
  input1 [label="{Input 1|{<i11>1|<i12>0|<i13>1}}"]
input2 [label="{Input 2|{<i21>1|<i22>1|<i23>1}}"]
  output [label="{{<o1>1|<o2>1|<o3>0|<o4>0}|Output}"]
  f1 [label="{{<f11>1|<f12>1|<f13>0}|Full-Adder|{<f1c>1|<f1s>0}}"]
  input1:i13 -> f1:f11 [label="1"]
  input2:i23 -> f1:f12 [label="1"]
  f1:f1s -> output:o4 [label="0"]
  f2 [label="{{<f21>0|<f22>1|<f23>1}|Full-Adder|{<f2c>1|<f2s>0}}"]
input1:i12 -> f2:f21 [label="0"]
  input2:i22 -> f2:f22 [label="1"]
  f1:f1c -> f2:f23 [label="1"]
  f2:f2s -> output:o3 [label="0"]
    f3 [label="{{<f31>1|<f32>1|<f33>1}|Full-Adder|{<f3c>1|<f3s>1}}"]
input1:i11 -> f3:f31 [label="1"]
  input2:i21 -> f3:f32 [label="1"]
  f2:f2c -> f3:f33 [label="1"]
  f3:f3s -> output:o2 [label="1"]
  f3:f3c -> output:o1 [label="1"]
   }
   \centering
   \caption{An Example Showin $101 +_{\mathbb{B}^{3}} 111 = 1100 $}
   \label{fig:RCA}
\end{figure}


\end{frame}

\begin{frame}{Ripple Adder Proof}
Every talk must include exactly $1$ proof.

\begin{proof}
  Induct on $n$. If $n=1$, we just have a full-adder. Otherwise, let $n = n + 1$.
  \begin{align}
    &\llbracket a_{n}\cdots a_{2}a_{1}a_{0} +_{\mathbb{B}^{n+1}}^{c_{0}} b_{n} \cdots b_{2}b_{1}b_{0} \rrbracket\\
    &= \llbracket (a_{n} \cdots a_{2}a_{1} +_{\mathbb{B}^{n}}^{c_{1}} b_{n} \cdots b_{2}b_{1})  r_{0}  \rrbracket \\
    &= 2\llbracket a_{n} \cdots a_{2}a_{1} +_{\mathbb{B}^{n}}^{c_{1}} b_{n} \cdots b_{2}b_{1} \rrbracket + \llbracket  r_{0}  \rrbracket \\
    &= 2 (\llbracket a_{n} \cdots a_{2}a_{1} \rrbracket  +  \llbracket b_{n-1} \cdots b_{2}b_{1} \rrbracket + \llbracket c_{1} \rrbracket ) + \llbracket  r_{0}  \rrbracket \\
    &= 2 \llbracket a_{n} \cdots a_{2}a_{1} \rrbracket  +  2\llbracket b_{n-1} \cdots b_{2}b_{1} \rrbracket + 2\llbracket c_{1} \rrbracket + \llbracket r_{0} \rrbracket \\
    &= 2 \llbracket a_{n} \cdots a_{2}a_{1} \rrbracket  +  2\llbracket b_{n-1} \cdots b_{2}b_{1} \rrbracket + \llbracket c_{1}r_{0} \rrbracket \\
    &= 2 \llbracket a_{n} \cdots a_{2}a_{1} \rrbracket  +  2\llbracket b_{n-1} \cdots b_{2}b_{1} \rrbracket + \llbracket +_{F}(a_{0}, b_{0}, c_{0}) \rrbracket \\
    &= 2 \llbracket a_{n} \cdots a_{2}a_{1} \rrbracket  +  2\llbracket b_{n-1} \cdots b_{2}b_{1} \rrbracket + \llbracket a_{0} \rrbracket + \llbracket b_{0}\rrbracket + \llbracket c_{0} \rrbracket \\
    &= 2 \llbracket a_{n} \cdots a_{2}a_{1} \rrbracket + \llbracket a_{0}  +  2\llbracket b_{n-1} \cdots b_{2}b_{1} \rrbracket + \llbracket b_{0} \rrbracket  + \llbracket c_{0} \rrbracket \\
    &= \llbracket a_{n} \cdots a_{2}a_{1}a_{0} \rrbracket + \llbracket b_{n} \cdots b_{2}b_{1}b_{0} \rrbracket + \llbracket c_{0} \rrbracket
\end{align}
\end{proof}
\end{frame}
\end{document}
