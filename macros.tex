\newcommand\nc\newcommand
\nc\rnc\renewcommand

%%%%%%%%%% AGDA ALIASES

\newcommand{\AP}{\AgdaPrimitive}
\newcommand{\APT}{\AgdaPrimitiveType}
\newcommand{\AK}{\AgdaKeyword}
\newcommand{\AM}{\AgdaModule}
\newcommand{\AS}{\AgdaSymbol}
\newcommand{\AStr}{\AgdaString}
\newcommand{\AN}{\AgdaNumber}
\newcommand{\AD}{\AgdaDatatype}
\newcommand{\AF}{\AgdaFunction}
\newcommand{\ARe}{\AgdaRecord}
\newcommand{\AFi}{\AgdaField}
\newcommand{\AB}{\AgdaBound}
\newcommand{\AIC}{\AgdaInductiveConstructor}
\newcommand{\AC}{\AgdaComment}
\newcommand{\AG}{\AgdaGeneralizable}
\newcommand{\APo}{\AgdaPostulate}
\newcommand{\AU}{\AgdaUnderscore}

\newcommand{\Set}{\mathbf{Set}}

\newcommand{\AR}{\AgdaRef}

%% Targets I've turned off
\newcommand\NoAgdaTarget[1]{}

\nc\uni[2]{\newunicodechar{#1}{\ensuremath{#2}}}

%% Remember to express horizontal movement in ems and vertical in exs and
%% nothing in pt, to adapt to font size changes.

\uni{⦃}{\{\hspace{-0.38em}\{}
\uni{⦄}{\}\hspace{-0.38em}\}}

\nc\sm[1]{\mathsf{#1}}
%% \uni{ℕ}{\sm{N}}
%% \uni{𝔹}{\sm{B}}
%% \uni{𝔽}{\sm{F}}

\usepackage{bm}
\uni{ℕ}{\mathbb{N}}
\uni{ℚ}{\mathbb{Q}}

\usepackage{mathrsfs}
\uni{𝒯}{\mathscr{T}}
%% \uni{𝒯}{\mathfrak{T}}
%% \uni{𝒯}{\mathcal{T}}

\uni{ℓ}{\ell}

\uni{⊗}{\otimes}
\uni{×}{\times}
\uni{∘}{\circ}
\uni{∙}{\bullet}
\uni{▵}{\vartriangle}
\uni{∎}{\blacksquare}

\usepackage{stmaryrd}

\uni{⇾}{\rightarrow}
\uni{⇨}{\rightarrowtriangle}
\uni{↝}{\leadsto}
\uni{↓}{\downarrow}
\uni{⇓}{\Downarrow}
\uni{↑}{\uparrow}

\uni{δ}{\delta}
\uni{λ}{\lambda}

\uni{≈}{\approx}
\uni{≡}{\equiv}
\uni{≗}{\circeq}
\uni{∃}{\exists}
\uni{∀}{\forall}
\uni{∈}{\in}
\uni{≤}{\le}
\uni{⊹}{\dotplus}
\uni{◇}{\diamond}
\uni{∅}{\varnothing}
\uni{∩}{\cap}
\uni{⊔}{\sqcup}
\uni{⊓}{\sqcap}
% \uni{⋯}{\cdots}
\uni{⋯}{\mathop{\cdot\cdot\cdot}}

\uni{∧}{\wedge}
\uni{∨}{\vee}
\uni{⊤}{\top}
\uni{⊥}{\bot}

%% Subscripts and superscripts
\nc\subs[1]{\ensuremath{_{\scriptscriptstyle #1}}}
\nc\sups[1]{\ensuremath{^{\scriptscriptstyle #1}}}

\uni{₀}{\subs 0}
\uni{₁}{\subs 1}
\uni{₂}{\subs 2}
\uni{₃}{\subs 3}
\uni{₄}{\subs 4}
\uni{₅}{\subs 5}
\uni{₆}{\subs 6}
\uni{₇}{\subs 7}
\uni{₈}{\subs 8}
\uni{₉}{\subs 9}

%% %%https://tex.stackexchange.com/questions/486120/using-the-unicode-prime-character-outside-math-mode-with-unicode-math-and-newuni
%% \AtBeginDocument{\newunicodechar{′}{\ensuremath{'}}}

\uni{′}{\ensuremath{'}}
\uni{″}{\ensuremath{'\hspace{-1pt}'}}

\usepackage{bbding}
\uni{✶}{\mbox{\tiny \SixStar}}

\uni{✢}{\raisebox{0.75pt}{\scalebox{0.65}{+}}}
\nc\carrot{\;\^{\;}}
\nc\inv{\ensuremath{^{-1}}}


%% To do: perhaps remove some substitutions below, and use the following.
\nc\SubStar[1]{\StrSubstitute{#1}{*}{\ast}}

\nc\ehat[1]{\ensuremath{\hat{#1}}}
\nc\edot[1]{\ensuremath{\dot{#1}}}
\nc\etilde[1]{\ensuremath{\tilde{#1}}}
%% \nc\etilde[1]{\~{#1}}

\usepackage{xstring}
\DeclareRobustCommand{\AgdaFormat}[2]{\IfStrEqCase{#1}{
  {\AU{}*\AU{}}{\_{\ast}\_}
  {0̇}{\edot{0}} {1̇}{\edot{1}}
  {+̇}{\edot{✢}}
  {*̇}{\edot{\ast}}
  {+}{✢}{\AU{}+̇\AU{}}{\_\edot{✢}\_}
  {*}{\ast}{\AU{}*̇\AU{}}{\_\edot{\ast}\_}
  {+̃}{\etilde{✢}}{\AU{}+̃\AU{}}{\_\etilde{✢}\_}
  {*̃}{\etilde{\ast}}{\AU{}*̃\AU{}}{\_\etilde{\ast}\_}
  {f̃}{\etilde{f}}{x̃}{\etilde{x}}
  {f̃⇃}{\etilde{\AFi{f}}}

{}{}}[#2]}


\nc\out[1]{}

%% \nc\noteOut[2]{\note{#1}\out{#2}}

%% To redefine for a non-draft
\nc\indraft[1]{#1}

%% I think \note gets defined by beamer.
\let\note\undefined

\definecolor{noteColor}{rgb}{0.5,0,0.3}

\nc\note[1]{\indraft{\textcolor{noteColor}{#1}}}

\nc\notefoot[1]{\note{\footnote{\note{#1}}}}

\nc\todo[1]{\note{To do: #1}}

\nc\eqnlabel[1]{\label{equation:#1}}
\nc\eqnref[1]{Equation~\ref{equation:#1}}
\nc\eqnreftwo[2]{Equations~\ref{equation:#1} and \ref{equation:#2}}

\nc\figlabel[1]{\label{fig:#1}}
\nc\figref[1]{Figure~\ref{fig:#1}}
\nc\figreftwo[2]{Figures~\ref{fig:#1} and \ref{fig:#2}}

\nc\seclabel[1]{\label{sec:#1}}
\nc\secref[1]{Section~\ref{sec:#1}}
\nc\secreftwo[2]{Sections~\ref{sec:#1} and~\ref{sec:#2}}
\nc\secrefs[2]{Sections \ref{sec:#1} through \ref{sec:#2}}

\nc\appref[1]{Appendix~\ref{sec:#1}}

%% The name \secdef is already taken
\nc\sectiondef[1]{\section{#1}\seclabel{#1}}
\nc\subsectiondef[1]{\subsection{#1}\seclabel{#1}}
\nc\subsubsectiondef[1]{\subsubsection{#1}\seclabel{#1}}

\nc\needcite{\note{[ref]}}

\nc\sectionl[1]{\section{#1}\seclabel{#1}}
\nc\subsectionl[1]{\subsection{#1}\seclabel{#1}}

\nc\workingHere{
\vspace{1ex}
\begin{center}
\setlength{\fboxsep}{3ex}
\setlength{\fboxrule}{4pt}
\huge\textcolor{red}{\framebox{Working here}}
\end{center}
\vspace{1ex}
}

%% For multiple footnotes at a point. Adapted to recognize \notefoot as well
%% as \footnote. See https://tex.stackexchange.com/a/71347,
\let\oldFootnote\footnote
\nc\nextToken\relax
\rnc\footnote[1]{%
    \oldFootnote{#1}\futurelet\nextToken\isFootnote}
\nc\footcomma[1]{\ifx#1\nextToken\textsuperscript{,}\fi}
\nc\isFootnote{%
    \footcomma\footnote
    \footcomma\notefoot
}

% Arguments: env, label, caption, body
\nc\figdefG[4]{\begin{#1}%[tbp]
\begin{center}
#4
\end{center}
\vspace{-1ex}
\caption{#3}
\figlabel{#2}
\end{#1}}

% Arguments: label, caption, body
\nc\figdef{\figdefG{figure}}
\nc\figdefwide{\figdefG{figure*}}

% Arguments: label, caption, body
\nc\figrefdef[3]{\figref{#1}\figdef{#1}{#2}{#3}}

\nc\picfile[1]{Figures/#1.pdf}

\nc\incpicW[2]{\includegraphics[width=#1\linewidth]{\picfile{#2}}}
\nc\incpic{\incpicW{0.7}}
%% \nc\incpic[1]{\includegraphics[width=0.7\linewidth]{\picfile{#1}}}

\nc\stdlibCitet[1]{\citet[\AM{#1}]{agda-stdlib}}
\nc\stdlibCitep[1]{\citep[\AM{#1}]{agda-stdlib}}
\nc\stdlibCite\stdlibCitep

%% https://tex.stackexchange.com/questions/94699/absolutely-definitely-preventing-page-break

\renewenvironment{samepage}
  {\par\nobreak\vfil\penalty0\vfilneg
   \vtop\bgroup}
  {\par\xdef\tpd{\the\prevdepth}\egroup
   \prevdepth=\tpd}

\nc\source{None}

\nc\twocolX[3]{\\[-1ex]
\begin{minipage}[c]{#1\linewidth}{#2}\end{minipage}
\begin{minipage}[c]{\linewidth}{#3}\end{minipage}\\[-1ex]}
\nc\twocol{\twocolX{0.45}}

\nc\threecolX[5]{\\[-1ex]
\begin{minipage}[c]{#1\linewidth}{#3}\end{minipage}
\begin{minipage}[c]{#2\linewidth}{#4}\end{minipage}
\begin{minipage}[c]{\linewidth}{#5}\end{minipage}\\[-1ex]}
\nc\threecol{\threecolX{0.3}{0.3}}

\nc\agda[1]{\ExecuteMetaData[\source.tex]{#1}}
\nc\agdaIn[2]{{\rnc\source{#1} \agda{#2}}}

\usepackage{amsthm}
\theoremstyle{plain}
\newtheorem{thm}{Theorem}
% \newtheorem{lemma}[thm]{Lemma}
\newtheorem{cor}[thm]{Corollary}
\newtheorem{conj}[thm]{Conjecture}
\newtheorem{prop}[thm]{Proposition}
\newtheorem{heur}[thm]{Heuristic}
\newtheorem{qn}[thm]{Question}
\newtheorem{claim}[thm]{Claim}

\theoremstyle{definition}
\newtheorem{defn}[thm]{Definition}
\newtheorem{cond}[thm]{Conditions}
\newtheorem*{notn}{Notation}

\theoremstyle{remark}
\newtheorem{rem}[thm]{Remark}
\newtheorem*{ex}{Example}
\newtheorem*{nonex}{Nonexample}
\newtheorem*{exer}{Exercise}



\author{Atticus Kuhn}
\title{Denotationally Correct Computer Arithmetic}
\date{\DTMnow{} GMT-8}
