Switching course, I will explore binary arithmetic at first.




# Imports
\begin{code}

{-# OPTIONS --allow-unsolved-metas #-}
open import Data.Nat.Base using (ℕ ; suc ; _+_ ; _^_ ; ⌊_/2⌋ )
open import Felix.Object using (Products ;  Coproducts ;   _×_ ; ⊤ ; _⊎_)
open import Felix.Raw using (Category ; Cartesian ; Cocartesian ; id ; _∘_ ; _▵_ ; twice ; _⊗_ ; inl ; inr ; ! ; _▿_ ; exl ; exr ; first ; second ; assocʳ ; assocˡ ;inAssocʳ′ ; transpose ;  dup ; swap ;   unitorⁱʳ ; constʳ ; inSwap; inAssocˡ′)
open import Relation.Binary.PropositionalEquality using (cong ; _≡_ ; subst)
open import Felix.Equiv using ( Equivalent)
import Felix.Laws as L using (Cartesian ; Category)
-- open import hasparity using (HasParity)
open import numbers using (Number)
-- open import euclideanDomain using (EuclideanDomain)
open import bitoperations using (HasBitOperations)
open import functor using (Functor)

\end{code}

# Module declaration and Instances
\begin{code}

module binary
       {lvl}
       { S : Set lvl }
     -- (A : Set)
       {{ps : Products S }}
       {{cps : Coproducts S }}
     {_⇨_ : S  → S  → Set} (let infix 0 _⇨_; _⇨_ = _⇨_)
     ⦃ x : Category _⇨_ ⦄ ⦃ y : Cartesian _⇨_ ⦄ ⦃ z : Cocartesian _⇨_ ⦄

         {{ equiv : Equivalent Agda.Primitive.lzero _⇨_ }}
         ⦃ cateq : L.Category _⇨_ ⦄
         ⦃ carteq : L.Cartesian _⇨_ ⦄
         (N : S)
         (Bit : S)
         {{bitops : HasBitOperations Bit}}
         {{parity : Number {{x = x}} N}}
         {{f : Functor {lvl = lvl} {S = S} ⦃ x = x ⦄ ⦃ ps = ps ⦄ ⦃ y = y ⦄ ⦃ equiv = equiv ⦄  ⦃ num = parity ⦄ }}
         -- ⦃ ed : EuclideanDomain N ⦄

     where

open HasBitOperations bitops
open Number parity
open Functor f
open Equivalent equiv
open L.Cartesian carteq
open L.Category cateq
-- open EuclideanDomain ed
-- Bit : S
-- Bit = N
\end{code}


# Declare Varaibles
\begin{code}
variable
  A B C D : S
  b : ℕ

\end{code}


# Define Bits
We will represent binary numbers as a list of bits
\begin{code}
List : ℕ → S → S
List 0 A = ⊤
List (suc n) A = A × (List n A)

Bits : ℕ → S
Bits n = List n Bit


\end{code}

# Convert Between Numbers and Little Endian Representations
Define the denotation function to convert between numbers and our binary representation of numbers. We are representing binary numbers as a little endian list of bits.
%<*Conversions>
\begin{code}

NatToLittleEndian : (b : ℕ) →  N ⇨ Bits b
NatToLittleEndian 0 =  !
NatToLittleEndian (suc b) = %2 ▵ (NatToLittleEndian b ∘ /2)


LittleEndianToNat : (b : ℕ) → Bits b ⇨ N
LittleEndianToNat  0 = p0
LittleEndianToNat (suc b) =   add ∘ ((if ∘ constʳ (p1 ▵ p0))  ⊗ (*2 ∘ LittleEndianToNat b ))
\end{code}
%</Conversions>

prove the the two functions are naturally isomorphic (they are inverses)

\begin{code}
--- todo
open ≈-Reasoning

-- ≈cong : {f g : A ⇨ B} → (evidence : f ≈ g) →
open import Data.Product using (_,_) renaming (_×_ to _×ₚ_)


!∘ : {A : S} {f : A ⇨ ⊤} → ! ∘ f ≈ !
!∘ = trans (trans (∀⊤) (sym ∀⊤)) (identityʳ)

ifconstr : {A B : S} {g h : ⊤ ⇨ A} {f : A ⇨ B} → f ∘ if ∘ constʳ (g ▵ h) ≈ if ∘ constʳ ((f ∘ g) ▵ (f ∘ h))
ifconstr {A} {B} {g} {h} {f} = {! !} -- begin
  -- f ∘ if ∘ ( _∘_  (first (g ▵ h))  (id ▵ !) )
  -- -- ≈⟨ ∘-assocˡ ⟩
  -- -- (f ∘ if) ∘ constʳ (g ▵ h)
  -- -- ≈⟨ ∘≈ˡ (ifdistrib) ⟩
  -- -- (f ∘ second (twice f)) ∘ constʳ (g ▵ h)
  -- -- ≈⟨ ∘-assocʳ ⟩
  -- -- f ∘ second (twice f) ∘ constʳ (g ▵ h)
  -- -- ≡⟨⟩
  -- -- f ∘ second (twice f) ∘ second (g ▵ h) ∘ unitorⁱʳ
  -- -- ≈⟨ ∘≈ˡ ∘-assocˡ ⟩
  -- -- f ∘ (second (twice f) ∘ second (g ▵ h)) ∘ unitorⁱʳ
  -- ≈⟨ {! !} ⟩
  -- if ∘ constʳ ((_∘_ f g) ▵ (f ∘ h))
  -- ∎
-- begin
 -- ! ∘ f
 -- ≈⟨ ∘≈ˡ (trans (∀⊤) (sym ∀⊤)) ⟩
 -- ! ∘ id
--  ≈⟨ identityʳ ⟩
 -- !
  -- ∎
postulate
  eq : {A B : S} {f g : A ⇨ B} →  f ≈ g →  f ≡ g


-- inverses2 : (b : ℕ) → (NatToLittleEndian b ∘ LittleEndianToNat b ≈ id)
-- inverses2 0 = trans ∀⊤ (sym ∀⊤)

inverses : (b : ℕ) → (NatToLittleEndian b ∘ LittleEndianToNat b ≈ id)
inverses 0 = trans ∀⊤ (sym ∀⊤)
inverses (suc b)  = begin
         NatToLittleEndian (suc b) ∘ LittleEndianToNat (suc b)
         ≡⟨⟩
          (%2 ▵ (NatToLittleEndian b ∘ /2)) ∘  (add ∘ ((if ∘ constʳ (p1 ▵ p0))  ⊗ (*2 ∘ LittleEndianToNat b )))
        ≈⟨ ▵∘ ⟩
          (%2 ∘  add ∘ ((if ∘ constʳ (p1 ▵ p0))  ⊗ (*2 ∘ LittleEndianToNat b )) ) ▵ ((NatToLittleEndian b ∘ /2) ∘  add ∘ ((if ∘ constʳ (p1 ▵ p0))  ⊗ (*2 ∘ LittleEndianToNat b )))
        ≈⟨ ▵≈ (begin
        %2 ∘  add ∘ ((if ∘ constʳ (p1 ▵ p0))  ⊗ (*2 ∘ LittleEndianToNat b ))
        ≈⟨ ∘-assocˡ ⟩
        (%2 ∘  add) ∘ ((if ∘ constʳ (p1 ▵ p0))  ⊗ (*2 ∘ LittleEndianToNat b ))
        ≈⟨ ∘≈ˡ (distrib) ⟩
        (  xor ∘ twice %2) ∘ ((if ∘ constʳ (p1 ▵ p0))  ⊗ (*2 ∘ LittleEndianToNat b ))
        ≈⟨ ∘-assocʳ ⟩
          xor ∘ twice %2 ∘ ((if ∘ constʳ (p1 ▵ p0))  ⊗ (*2 ∘ LittleEndianToNat b ))
        ≈⟨ ∘≈ʳ ⊗∘⊗ ⟩
          xor ∘  ((%2 ∘ if ∘ constʳ (p1 ▵ p0))  ⊗ (%2 ∘ *2 ∘ LittleEndianToNat b ))

        ≈⟨ ∘≈ʳ (⊗≈ (begin
        %2 ∘ if ∘ constʳ (p1 ▵ p0)
        ≈⟨ ∘-assocˡ ⟩
        (%2 ∘ if) ∘ constʳ (p1 ▵ p0)
        ≈⟨ ∘≈ˡ (ifdistrib) ⟩
        (if ∘ second (twice %2)) ∘ constʳ (p1 ▵ p0)
        ≈⟨ ∘-assocʳ ⟩
        if ∘ second (twice %2) ∘ constʳ (p1 ▵ p0)
        ≈⟨ ∘≈ʳ ∘-assocˡ ⟩
        if ∘ (second (twice %2) ∘  second (p1 ▵ p0)) ∘ unitorⁱʳ
        ≈⟨ ∘≈ʳ (∘≈ˡ (second∘second)) ⟩
        if ∘ (second (twice %2 ∘ (p1 ▵ p0))) ∘ (id ▵ !)
        ≈⟨ ∘≈ʳ (second∘▵) ⟩
        if ∘ (id ▵ (twice %2 ∘ (p1 ▵ p0)) ∘ !  )
        ≈⟨ ∘≈ʳ (▵≈ʳ (∘≈ˡ (⊗∘▵))) ⟩
        if ∘ (id ▵ ( ( (%2 ∘ p1) ▵ (%2 ∘ p0))) ∘ !  )
        ≈⟨ ∘≈ʳ (▵≈ʳ (∘≈ˡ (▵≈ oneodd zeroeven ))) ⟩
        if ∘ (id ▵ ( ( bit1 ▵ bit0) ∘ !  ))
        ≈⟨ ∘≈ʳ {!!} ⟩
        if ∘ constʳ (bit1 ▵ bit0)
        ∎
        ){!!}) ⟩
        xor ∘ ( (if ∘ constʳ (bit1 ▵ bit0)) ⊗  (bit0 ∘ ! ))
        ≈⟨ {! !} ⟩
        exl ∘ id
        ∎
        )

        (begin
          (NatToLittleEndian b ∘ /2) ∘  add ∘ ((if ∘ constʳ (p1 ▵ p0))  ⊗ (*2 ∘ LittleEndianToNat b ))
          ≈⟨  assoc ⟩
          NatToLittleEndian b ∘ (/2 ∘  (add ∘ ((if ∘ constʳ (p1 ▵ p0))  ⊗ (*2 ∘ LittleEndianToNat b ))))
          ≈⟨  ∘≈ʳ ∘-assocˡ ⟩
          NatToLittleEndian b ∘ ((/2 ∘  add) ∘ ((if ∘ constʳ (p1 ▵ p0))  ⊗ (*2 ∘ LittleEndianToNat b )))
          ≈⟨ ∘≈ʳ (∘≈ˡ push)⟩
          NatToLittleEndian b ∘ ((add ∘ twice /2) ∘ ((if ∘ constʳ (p1 ▵ p0))  ⊗ (*2 ∘ LittleEndianToNat b )))
          ≈⟨ ∘≈ʳ ∘-assocʳ ⟩
          NatToLittleEndian b ∘ (add ∘ (twice /2 ∘ ((if ∘ constʳ (p1 ▵ p0))  ⊗ (*2 ∘ LittleEndianToNat b ))))
          ≈⟨ ∘≈ʳ (∘≈ʳ ⊗∘⊗) ⟩
          NatToLittleEndian b ∘ (add ∘ ( ( (/2 ∘ (if ∘ constʳ (p1 ▵ p0)))  ⊗ ( /2 ∘ *2 ∘ LittleEndianToNat b ))))
          ≈⟨ ∘≈ʳ (∘≈ʳ (⊗≈
            (begin
            /2 ∘ if ∘ constʳ (p1 ▵ p0)
            ≈⟨ ∘-assocˡ ⟩
            (/2 ∘ if) ∘ constʳ (p1 ▵ p0)
            ≈⟨ ∘≈ˡ ifdistrib ⟩
            (if ∘ second (twice /2)) ∘ constʳ (p1 ▵ p0)
            ≈⟨ ∘-assocʳ ⟩
            if ∘ second (twice /2) ∘ (second (p1 ▵ p0) ∘ (id ▵ !))
            ≈⟨ ∘≈ʳ ∘-assocˡ ⟩
             if ∘ (second (twice /2) ∘ second (p1 ▵ p0)) ∘ (id ▵ !)
             ≈⟨ ∘≈ʳ (∘≈ˡ second∘second) ⟩
            -- ≈⟨ ∘≈ʳ (∘≈ˡ (second∘second)) ⟩
            if ∘ second (twice /2 ∘ (p1 ▵ p0)) ∘ (id ▵ !)
            ≈⟨ ∘≈ʳ (second∘▵) ⟩
            if ∘ (id ▵ ((twice /2 ∘ (p1 ▵ p0)) ∘ !))
            ≈⟨ ∘≈ʳ (▵≈ʳ (∘≈ˡ ⊗∘▵)) ⟩
            if ∘ (id ▵ (( (( /2 ∘ p1 ) ▵ ( /2 ∘ p0 ))) ∘ !))
            ≈⟨ ∘≈ʳ (▵≈ʳ (∘≈ˡ (▵≈ (p1/2) (p0/2)))) ⟩
            if ∘ (id ▵ (( ( p0  ▵ p0) ∘ !)))
            ≈⟨ ∘≈ʳ (▵≈ʳ (▵∘)) ⟩ -- ▵∘
            if ∘ (id ▵ (( ( (p0 ∘ !)  ▵ (p0 ∘ !)) )))
            -- ≈⟨ ∘≈ʳ (▵≈ʳ ( ▵≈ (∘≈ʳ (sym (∀⊤))) (∘≈ʳ (sym (∀⊤))) )) ⟩
            -- if ∘ (id ▵ (( ( (p0  ∘ exl)  ▵ (p0 ∘  exr)) )))
            -- ≈⟨ ∘≈ʳ (▵≈ʳ (▵≈ {! !}  {! !} ) )  ⟩
            -- if ∘ (id ▵ (( ( (p0 ∘ exl)  ▵ (p0 ∘ exr)) )))
            -- ≈⟨ sym (ifdistrib) ⟩
            -- (p0 ∘ !) ∘ if
            ≈⟨ {! !} ⟩
            -- ≈⟨ ∘≈ʳ (▵≈ʳ ∘-assocˡ) ⟩
            -- if ∘ (id ▵ ((twice /2 ∘ (p1 ▵ p0)) ∘ !))
            -- ≈⟨ ∘-assocʳ ⟩
            -- p0 ∘ ! ∘ if
            -- ≈⟨ ∘≈ʳ (!∘ ) ⟩
             p0 ∘ !
            ∎
            )
            (begin
                 /2 ∘ *2 ∘ LittleEndianToNat b
               ≈⟨ ∘-assocˡ ⟩
               (/2 ∘ *2) ∘ LittleEndianToNat b
               ≈⟨ ∘≈ˡ /2∘*2 ⟩
               id ∘ LittleEndianToNat b
               ≈⟨ identityˡ ⟩
               LittleEndianToNat b
             ∎
          )  ) ) ⟩
          NatToLittleEndian b ∘ (add ∘ (( p0 ∘ !)  ⊗  LittleEndianToNat b))
          ≈⟨ {! ∘≈ʳ addIdLeft!} ⟩

          NatToLittleEndian b ∘   LittleEndianToNat b ∘ exr
          ≈⟨ ∘-assocˡ ⟩

          (NatToLittleEndian b ∘   LittleEndianToNat b) ∘ exr
          ≈⟨ ∘≈ˡ (inverses b) ⟩
          id ∘ exr
          ≈⟨ identityˡ ⟩
          exr
          ≈⟨ sym (identityʳ ) ⟩
          exr ∘ id
        ∎
        )⟩
        (exl ∘ id) ▵ (exr ∘ id)
        ≈⟨ sym (  ∀×← ( refl , refl ) ) ⟩
        id
  ∎
\end{code}

# Adder

Define a half adder.

Note that for convention, the left output is the sum and the right output is the carry.
%<*HalfAdder>
\begin{code}
halfAdder : Bit × Bit  ⇨ Bit × Bit
halfAdder  = xor ▵ and
\end{code}
%</HalfAdder>
Define a ful adder.

Note that for convention, the left output is the sum and the right output is the carry.
%<*FullAdder>
\begin{code}
fullAdder : (Bit × Bit) × Bit ⇨ Bit × Bit
fullAdder =   first or ∘ inSwap (inAssocˡ′ (first halfAdder)) ∘  first halfAdder
\end{code}
%</FullAdder>

Specification that half adder and full adder are correct.

The specification for the half adder says that if I add two bits, that should be the same as putting the two bits through a half-adder and then converting the result to a number.

The specification for the full adder says that if I have any three bits, adding them should be the same as inputting them to the full-adder and then converting the result to a number.

%<*HAS>
\begin{code}
halfAdderSpecification :   second exl ∘  (NatToLittleEndian 2) ∘ add ≈ halfAdder ∘ twice %2
halfAdderSpecification = trans (begin
  second exl ∘ NatToLittleEndian 2 ∘ add
  ≈⟨ ∘-assocˡ ⟩
  (second exl ∘ NatToLittleEndian 2) ∘ add
  ≈⟨ ∘≈ˡ (second∘▵) ⟩
  ( %2 ▵ ( exl ∘ ( %2 ▵ ( ! ∘ /2 ) ) ∘ /2 )  ) ∘ add
  ≈⟨ ∘≈ˡ (▵≈ʳ ∘-assocˡ ) ⟩
  ( %2 ▵ ( (exl ∘ ( %2 ▵ ( ! ∘ /2 ) )) ∘ /2 )  ) ∘ add
  ≈⟨ ∘≈ˡ (▵≈ʳ (∘≈ˡ exl∘▵) ) ⟩
  ( %2 ▵ (  %2 ∘ /2 )  ) ∘ add
  ≈⟨ ▵∘ ⟩
  (%2 ∘ add) ▵ (  (%2 ∘ /2 ) ∘ add  )
  ≈⟨ ▵≈ʳ ∘-assocʳ ⟩
   (%2 ∘ add) ▵ (%2 ∘ /2 ∘ add)
  ∎) (sym (begin
  halfAdder ∘ twice %2
  ≈⟨ ▵∘ ⟩
  (xor ∘ twice %2) ▵ (and ∘ twice %2)
  ≈⟨ ▵≈ ( sym (distrib) ) distrib/2 ⟩
   (%2 ∘ add) ▵ (%2 ∘ /2 ∘ add)
  ∎))
\end{code}
%</HAS>

\begin{code}
fullAdderSpecification :  fullAdder ∘ (twice %2 ⊗ %2)  ≈ second exl ∘  (NatToLittleEndian 2) ∘ add ∘ first add
fullAdderSpecification =  {! !}

\end{code}

Define some utility functions
\begin{code}
scoop :  (A × B) × (C × D) ⇨ ((A × B) × D) × C
scoop =  ((exl ∘ exl ▵ exr ∘ exl) ▵ ( exr ∘ exr)) ▵ (exl ∘ exr)


blender :  (A × B) × C ⇨ (A × C) × B
blender = inAssocʳ′ (second (swap))
\end{code}

Use ripple addition to add two `b` bit little endian representations of binary numbers.
\begin{code}

zip : (length : ℕ) → List length A × List length B ⇨ List length (A × B)
zip 0 = !
zip (suc length) =  second (zip length) ∘ transpose

genericRipple : (length : ℕ) → (A × B) × C ⇨ D × C → (List length (A × B)) × C ⇨ (List length D) × C
genericRipple 0 f = ! ⊗ id
genericRipple (suc length) f = inAssocʳ′ (second (genericRipple length f ∘ swap))  ∘ first f ∘ inAssocʳ′ (second swap)


rippleAddLittleEndian : (b : ℕ) → ((Bits b) × (Bits b)) × Bit ⇨ (Bits b) × Bit
rippleAddLittleEndian b = genericRipple b fullAdder ∘ first (zip b)
\end{code}
Make a variation with no carry bit

\begin{code}

-- snoc is "cons" backwards and appends an element to the end of a list
snoc : (length : ℕ) → List length A × A ⇨ List (suc length) A
snoc 0 = swap
snoc (suc length) = second (snoc length) ∘ assocʳ
rippleAddLittleEndianNoCarry : (b : ℕ) → (Bits b) × (Bits b) ⇨ (Bits (suc b))
rippleAddLittleEndianNoCarry b  = (snoc b) ∘ (rippleAddLittleEndian b) ∘ constʳ bit0
\end{code}

Specification of ripple addition

\begin{code}
rippleAddSpecification : (b : ℕ ) → LittleEndianToNat (suc b) ∘ rippleAddLittleEndianNoCarry b ≈ add ∘ twice (LittleEndianToNat b)
rippleAddSpecification b = {!  !}

\end{code}
